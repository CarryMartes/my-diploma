\chapter{Introduction}\label{ch:intro}
%these sections are optional, up-to the author
\section{Motivation}
We have been thinking about the topic of our thesis for a very long time. But as soon as we saw the topics that our coach send to us, without hesitation, we decided to take the topic "{\mytitle}". The thing is that even when we were in college, we thought about a platform that would make it easier for teachers and at the same time be very easy to use for students. This idea has been in our heads for a long time. Fortunately for us, all the cards turned out just so that we finally took on this project.
\section{Aims and Objectives}
The main goal of our project was: Help teachers in the most important and difficult part of the semester - grading. Since there are so many students, grading takes a very long time. But with our project, teachers will be able to grade a hundred times faster, thus saving a lot of time. Since there are many different platforms that help teachers evaluate students, we decided to automate everything and combine a set of all platforms in one.
\section{Thesis Outline}
As mentioned earlier, we realized that there are many platforms that are similar to our idea. Therefore, we had to come up with something innovative and new. Besides the fact that we wanted to create something completely new, we decided to combine our idea with the favorite platform of all programmers, GitHub. We did this because we understood that most IT-related projects are contained in GitHub, and so that the project does not have to be resubmitted a million times - the teacher can evaluate the student’s work, as they say “without leaving the cash register”. That is why we decided to integrate our idea with Git. Our vision for the project was based on several advanced platforms such as: Moodle, Google Classroom, SDU Portal. Based on them, we added something new, improved existing functions, or completely removed some functions that, in our opinion, did not particularly fit the concept of our project.